% !TEX root = ./main_TCS.tex

%% PACKAGES
%\usepackage[usenames,dvipsnames]{color}
\usepackage[normalem]{ulem}
\usepackage{amsmath}
%\usepackage{amsthm} 
%\usepackage{amsfonts}
\usepackage{amssymb}
\usepackage{stmaryrd}
\usepackage{subcaption}
%\usepackage{graphicx} 
 \usepackage[all]{xy}
\usepackage{bussproofs}
\usepackage{mathtools} 
\usepackage{wasysym}
\usepackage{lscape}
\usepackage{graphicx}
\usepackage{cancel}
\newcommand{\plusarrow}{\xrightarrow{+\Delta V \%}}

\newcommand{\CNA}{\textsf{CNA}}

%% LINK CALCULUS
\newcommand{\link}[2]{\ensuremath{{}^{#1} \backslash_{#2}^{}}}
\newcommand{\chainedlink}[2]{\ensuremath{\backslash_{#1}^{#2}}}
\newcommand{\startchain}[1]{\ensuremath{{}^{#1}_{}}}
\newcommand{\chainend}[1]{\ensuremath{\backslash_{#1}^{}}}
%\newcommand{\midlink}[2]{\ensuremath{{}_{#1} {}^{#2}}}
\newcommand{\midlink}[2]{\ensuremath{{}_{#1}^{#2}}}
\newcommand{\silent}{\tau}
%\newcommand{\noact}{*}
%\newcommand{\noact}{_{\square}}
\newcommand{\noact}{\scriptstyle {\square}}
\newcommand{\noacts}{\scriptstyle \square}
\newcommand{\merges}[2]{\ensuremath{#1 \bullet   #2}}
%\newcommand{\merges}[2]{\ensuremath{#1 \,\wr\, #2}}
\newcommand{\restrict}[1]{\ensuremath{(\nu\, #1)}}
\newcommand{\recproc}[2]{\ensuremath{\mathsf{rec}\ #1.\, #2}}
\newcommand{\nil}{\mathbf{0}}
%\newcommand{\solid}[1]{\ensuremath{\{\, #1\,\}}}
\newcommand{\tuple}[1]{\ensuremath{\langle #1\rangle}}
\newcommand{\variable}[1]{\ensuremath{\underline{#1}}}
%\newcommand{\subst}[2]{#2 \mapsto #1}
\newcommand{\subst}[2]{#1/#2}
\newcommand{\ground}[1]{\ensuremath{\llbracket\, #1 \,\rrbracket}}
\newcommand{\values}[1]{\ensuremath{\mathit{vals}(#1)}}
\newcommand{\variables}[1]{\ensuremath{\mathit{vars}(#1)}}
\newcommand{\extruded}[1]{\ensuremath{\mathit{ex}(#1)}}
\newcommand{\innames}[1]{\ensuremath{\mathit{in}(#1)}}
\newcommand{\outnames}[1]{\ensuremath{\mathit{out}(#1)}}
\newcommand{\names}[1]{\ensuremath{\mathit{names}(#1)}}
\newcommand{\forwarded}[1]{\ensuremath{\llparenthesis #1\rrparenthesis}}
\newcommand{\instanceof}{\preceq}
\newcommand{\elc}{\ensuremath{\epsilon}}
\newcommand{\reduce}[1]{\ensuremath{\mathsf{e}(#1)}}
%\newcommand{\stretcheq}{\ensuremath{\bowtie}}
\newcommand{\stretcheq}{\scriptsize \mathrel \vartriangleright \joinrel \mathrel \vartriangleleft}
%$\mathrel \vartriangleright \joinrel \mathrel \vartriangleleft$
%\newcommand{\blackstretcheq}{\ensuremath{\fbowtie}}
%\newcommand{\llink}{\ensuremath{\mathrel \square \joinrel \mathrel \square}}
\newcommand{\blackstretcheq}{\mathrel{\scriptsize {\mathrel \blacktriangleright \joinrel \mathrel \blacktriangleleft}}}
%$\mathrel \blacktriangleright \joinrel \mathrel \blacktriangleleft$
\newcommand{\simnet}{\ensuremath{\stackrel{\stretcheq}{\sim}}}
\newcommand{\simbnet}{\ensuremath{\stackrel{\blackstretcheq}{\sim}}}

\newcommand{\solid}[1]{{#1}\downarrow}
\newcommand{\notsolid}[1]{{#1}\ndownarrow}

\newcommand{\halfl}[1]{{}^{#1\backslash}}
\newcommand{\halfr}[1]{{}_{\backslash #1}}
\newcommand{\blockchain}[3]{\ensuremath{\left({\big \backslash\!\!\big \backslash}\ {}_{#1}^{\noact}\chainedlink{\noact}{#2}}\right)_{\tiny\begin{array}{l}\\[2pt] \hspace{-42pt}#3\end{array}}\hspace{-0.3cm}}

%% GENERAL
\newcommand{\bn}[1]{\ensuremath{\mathit{bn}(#1)}}
\newcommand{\fn}[1]{\ensuremath{\mathit{fn}(#1)}}
\newcommand{\prodsep}{\;\mid\;}
\newcommand{\defeq}{\triangleq}
\newcommand{\setof}[1]{\{\, #1 \,\}}
\newcommand{\setofmid}[2]{\{\, #1\,\mid\, #2 \,\}}

%% AMBIENT
\newcommand{\nilMA}{\nil}
\newcommand{\incap}[1]{\ensuremath{\mathsf{in}\, #1}}
\newcommand{\outcap}[1]{\ensuremath{\mathsf{out}\, #1}}
\newcommand{\opencap}[1]{\ensuremath{\mathsf{open}\, #1}}
\newcommand{\inprefix}[2]{\ensuremath{\incap{#1}.#2}}
\newcommand{\outprefix}[2]{\ensuremath{\outcap{#1}.#2}}
\newcommand{\openprefix}[2]{\ensuremath{\opencap{#1}.#2}}
\newcommand{\ambient}[2]{\ensuremath{#1 [\, #2\,]}}
\newcommand{\replicated}[1]{\ensuremath{\mathsf{!} #1}}

%% ENCODING
\newcommand{\CNAencode}[2]{\ensuremath{\llbracket\, #1 \,\rrbracket_{#2}}}
\newcommand{\ttop}{{\tt t}}

%% ??
\newcommand{\In}{\, {\cal I}n \, }
\newcommand{\Nin}{\, \not \!\!\!\!{\cal I}n \, }


%%Assertion
\newcommand{\entails}{\models}
