% !TEX root = ./main_TCS.tex

\section{Conclusion}\label{sec:conclusion}

In this paper we have introduced c\CNA, that generalises \CNA\ by allowing the use of prefixes that are link chains and not just single links.
This extension was initially described in the future work section of~\cite{BBB17}.
Thanks to this enhancement, c\CNA\ allowed us to define 
%shown 
a faithful encoding of 
Reaction Systems in a process algebraic framework.  
This encoding
%translation 
shows several benefits.
First, contexts of RSs can be easily defined recursively and exhibit non deterministic behaviour.
Second, the operational semantics is defined in a compositional way by a set of SOS inference rules.
Third, we have defined a new assertion language, which allows us to specify
the properties to be verified over the labels of the operational semantics.
Assertions can be used to tailor the classical notion of bisimilarity and Hennessy-Milner logic to focus on some particular aspects or experiments. We have called \emph{bio-similarity} the induced notion of equivalence.

We are currently investigating how to integrate our methodology
with other formal techniques to prove 
properties of the modeled systems, along the lines in~\cite{CFHOT15,OCHF16,BBGLBH2017}.
Moreover, we are considering possible enhancements of RSs based on entity mutation and on the possibility for two RSs to exchange entities.

As future work, we plan to implement a prototype of our 
%framework,
embedding,
with an automatic translation from RSs to  {\tt link}-calculus, so to exploit the implementation of  the symbolic semantics of 
the {\tt link}-calculus~\cite{BrodoO17} that can be found in~\cite{tool}.

