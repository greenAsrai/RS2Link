% !TEX root =  main_RS2L_LNCS.tex

%Natural computing is an emerging area of research for modeling biological 
%computations by means of computer science formal methods. Reaction systems 
%have been defined in this area as a qualitative abstraction inspired by the 
%functioning of living cells, suitable to model the main
%mechanisms of biochemical reactions.
%Natural computing is an emerging area of research for modeling biological 
%computations by means of computer science formal methods. 
In the area of Natural Computing, reaction systems are
a qualitative abstraction inspired by the 
functioning of living cells, suitable to model the main
mechanisms of biochemical reactions.
This model has already been applied
and extended successfully to various areas of research. Reaction 
systems interact with the environment represented by the context, and
pose problems of implementation, as it is a new computation model. 
In this paper we consider the {\tt link}-calculus, which allows to model 
multiparty interaction in concurrent systems, and show that it allows to
embed reaction systems, by representing the behaviour of each entity and
preserving faithfully their features. 
We show the correctness and completeness 
of our embedding.
We illustrate our framework by showing the embedding of 
a few examples expressing computer science
and biological applications.
%how to embed a \emph{lac} operon 
%regulatory network. 
We then modify the classical notion (in concurrency) of bisimulation
making it more suitable for proving properties of reaction systems, in our
framework.
We define a new assertion language, which allows to specify
the properties to be verified, and show that it extends Hennessy-Milner logic
to our framework.
Finally, our methodology can contribute to increase the expressiveness
of reaction systems, by exploiting the interaction among 
different reaction systems.
%entities and the context. 
