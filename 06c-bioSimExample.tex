% !TEX root = ./main_RS2L_LNCS.tex
\subsection{Bio-simulation at work}

%By using a simple example, 
We will show how bio-simulation works. For the sake of space, we consider a  very simple
 example with only two reactions. 
%Section~\ref{subsec:toy}, with only two reactions:
Reaction $P_1$ requires $G$  to produce $C$; reaction $P_2$
requires $C$ to produce $G$; 
both reactions have $H$ as inhibitor.
%here we do not consider inhibitors.  
Now, we set two systems defined by the same two reactions, with the two different initial configuration, and with two different context definitions.
The two reactions work as follows (where we omit to specify the cases where $H$ is present, as they will never happen):
\[
\begin{array}{lcl}
P_1 & \defeq &\startchain{\silent}\chainedlink{G_i}{\noact}\chainedlink{\noact}{G_o}
                                                  \chainedlink{\overline{H}_i}{\noact}\chainedlink{\noact}{\overline{H}_o}
					        \chainedlink{r_{2}}{\noact}\chainedlink{\noact}{p_1}
					        \chainedlink{\widetilde{C}_i}{\noact}\chainedlink{\noact}{\widetilde{C}_o}
			\chainend{p_{2}}.P_1  \   \
%			 +\ 
%					\ \  \startchain{\silent}\chainedlink{H_i}{\noact}\chainedlink{\noact}{H_o}\chainedlink{r_{2}}{\noact}\chainedlink{\noact}{p_1} \chainend{p_{2}}.P_1
			 +\ 
					\ \  \startchain{\silent}\chainedlink{\overline{G}_i}{\noact}\chainedlink{\noact}{\overline{G}_o}\chainedlink{r_{2}}{\noact}\chainedlink{\noact}{p_1} \chainend{p_{2}}.P_1\\
					P_2 & \defeq &\startchain{r_2}\chainedlink{C_i}{\noact}\chainedlink{\noact}{C_o}
                                                  \chainedlink{\overline{H}_i}{\noact}\chainedlink{\noact}{\overline{H}_o}
					        \chainedlink{\mathit{cxt}}{\noact}\chainedlink{\noact}{p_2}
					        \chainedlink{\widetilde{G}_i}{\noact}\chainedlink{\noact}{\widetilde{G}_o}
			\chainend{\silent}.P_2  \   \
%			 +\ 
%					\ \  \startchain{r_2}\chainedlink{H_i}{\noact}\chainedlink{\noact}{H_o}\chainedlink{\mathit{cxt}}{\noact}\chainedlink{\noact}{p_2} \chainend{\silent}.P_2
			 +\ 
					\ \  \startchain{r_2}\chainedlink{\overline{C}_i}{\noact}\chainedlink{\noact}{\overline{C}_o}\chainedlink{\mathit{cxt}}{\noact}\chainedlink{\noact}{p_2} \chainend{\silent}.P_2
			\end{array}
\]
The two contexts follow :\\
\[
\begin{array}{l@{\hspace{0.1cm}}c@{\hspace{0.1cm}}l}
\mathit{Cxt}_1 & \defeq &\startchain{\mathit{cxt}}\chainedlink{\widehat{C}_i}{\noact}\chainedlink{\noact}{\widehat{C}_o}\chainedlink{\underline{G}_i}{\noact}\chainedlink{\noact}{\underline{G}_o}
\chainedlink{\underline{H}_i}{\noact}\chainedlink{\noact}{\underline{H}_o}\chainend{p_1}.\mathit{Cxt}_1 			                      +
                              \startchain{\mathit{cxt}}\chainedlink{\underline{C}_i}{\noact}\chainedlink{\noact}{\underline{C}_o}\chainedlink{\underline{G}_i}{\noact}\chainedlink{\noact}{\underline{G}_o}\chainedlink{\underline{H}_i}{\noact}\chainedlink{\noact}{\underline{H}_o}\chainend{p_1}.\mathit{Cxt}_1 \\
                              \\
                              \mathit{Cxt}_2 & \defeq &\startchain{\mathit{cxt}}\chainedlink{\widehat{G}_i}{\noact}\chainedlink{\noact}{\widehat{G}_o}\chainedlink{\underline{C}_i}{\noact}\chainedlink{\noact}{\underline{C}_o}\chainedlink{\underline{H}_i}{\noact}\chainedlink{\noact}{\underline{H}_o}\chainend{p_1}.\mathit{Cxt}_2 	                      +
                              \startchain{\mathit{cxt}}\chainedlink{\underline{G}_i}{\noact}\chainedlink{\noact}{\underline{G}_o}\chainedlink{\underline{C}_i}{\noact}\chainedlink{\noact}{\underline{C}_o}\chainedlink{\underline{H}_i}{\noact}\chainedlink{\noact}{\underline{H}_o}\chainend{p_1}.\mathit{Cxt}_2
\end{array}
\]

The definition of the processes encoding $G$ and $C$ is similar, and it is given in a parametric way:
\[
\begin{array}{lcl @{\hspace{1cm}}lcl}
P(G) &\defeq&\link{G_i}{G_o}.\overline{P}(G)\   & \overline{P}(G)&\defeq&\link{\overline{G}_i}{\overline{G}_o} \link{\noact}{\noact}\link{\widetilde{G}_i}{\widetilde{G}_o}P(G) \ \ 			                      
\end{array}
\]
and we have $P_G \defeq P(G)$, $P_C \defeq P(C)$,  then $\overline{P}_H \defeq \link{\overline{H}_i}{\overline{H}_o}.\overline{P}_H + \startchain{\overline{H}_i}\chainedlink{\overline{H}_o}{\noact}\chainedlink{\noact}{\overline{H}_i}\chainend{\overline{H}_i}.\overline{P}_H$, as $H$ is neither produced nor provided by the context.
Then, the initial configuration of system $Sys_1$ includes $C$ and not $G$ and the context can only provide $G$,
the initial configuration of system $Sys_2$ includes $G$ and not $C$ and the context can only provide $C$:
\[
\begin{array}{lcl}
\mathit{Sys}_1 & \defeq & I\mid  P_1 \mid P_2 \mid P_G \mid \overline{P}_C \mid \overline{P}_H\mid \mathit{Cxt}_1 \\
\mathit{Sys}_2 & \defeq & I \mid P_1 \mid P_2 \mid \overline{P}_G \mid P_C \mid \overline{P}_H\mid \mathit{Cxt}_2\\
\end{array}
\]

Fig.~\ref{fig:ltss} shows the labelled transition system of $Sys_1$, with initial state only  containing  $G$,  and the transition system of $Sys_2$, with initial state only containing $C$, limited to the complete
and solid labels. As before we show the output of the $\mathit{flat}(\cdot)$ function applied to the transition labels:
%where

{\footnotesize \[
\begin{array}{lcl@{\hspace{0.3cm}} lcl}
%\
\ell_1 &\defeq & 
%\startchain{\silent}\chainedlink{G_i}{\bf G_i}\chainedlink{\bf G_o}{ G_o}\chainedlink{r_2}{r_2}\chainedlink{\overline{C}_i}{\bf \overline{C}_i}\chainedlink{\bf \overline{C}_o}{\overline{C}_o}
%\chainedlink{\mathit{cxt}}{\mathit{cxt}}\chainedlink{\widehat{C}_i}{\bf \widehat{C}_i}\chainedlink{\bf \widehat{C}_o}{\widehat{C}_o}
%\chainedlink{p_1}{p_1}\chainedlink{\widetilde{C}_i}{\bf \widetilde{C}_i}\chainedlink{\bf \widetilde{C}_o}{\widetilde{C}_o}\chainedlink{p_2}{p_2}\chainend{\silent}
r_1~G~\overline{H}~r_2~\overline{C}~\mathit{cxt}~\widehat{C}~\underline{G}~\underline{H}~p_1~\widetilde{C}~p_2
&
\ell'_1 &\defeq & 
%\startchain{\silent}\chainedlink{G_i}{\bf G_i}\chainedlink{\bf G_o}{ G_o}\chainedlink{r_2}{r_2}\chainedlink{\overline{C}_i}{\bf \overline{C}_i}\chainedlink{\bf \overline{C}_o}{\overline{C}_o}
%\chainedlink{\mathit{cxt}}{\mathit{cxt}}\chainedlink{\underline{C}_i}{\bf \underline{C}_i}\chainedlink{\bf \underline{C}_o}{\underline{C}_o}
%\chainedlink{p_1}{p_1}\chainedlink{\widetilde{C}_i}{\bf \widetilde{C}_i}\chainedlink{\bf \widetilde{C}_o}{\widetilde{C}_o}\chainedlink{p_2}{p_2}\chainend{\silent}
r_1~G~\overline{H}~r_2~\overline{C}~\mathit{cxt}~\underline{C}~\underline{G}~\underline{H}~p_1~\widetilde{C}~p_2\\
\ell_2 &\defeq &
% \startchain{\silent}\chainedlink{\overline{G}_i}{\bf \overline{G}_i}\chainedlink{\bf \overline{G}_o}{ \overline{G}_o}\chainedlink{r_2}{r_2}\chainedlink{C_i}{\bf C_i}\chainedlink{\bf C_o}{C_o}
%\chainedlink{\mathit{cxt}}{\mathit{cxt}}\chainedlink{\widehat{C}_i}{\bf \widehat{C}_i}\chainedlink{\bf \widehat{C}_o}{\widehat{C}_o}
%\chainedlink{p_1}{p_1}\chainedlink{p_2}{p_2}\chainedlink{\widetilde{G}_i}{\bf \widetilde{G}_i}\chainedlink{\bf \widetilde{G}_o}{\widetilde{G}_o}\chainend{\silent}
r_1~\overline{G}~r_2~C~\overline{H}~\mathit{cxt}~\widehat{C}~\underline{G}~\underline{H}~p_1~p_2~\widetilde{G}
&
\ell'_2 &\defeq & 
% \startchain{\silent}\chainedlink{\overline{G}_i}{\bf \overline{G}_i}\chainedlink{\bf \overline{G}_o}{ \overline{G}_o}\chainedlink{r_2}{r_2}\chainedlink{C_i}{\bf C_i}\chainedlink{\bf C_o}{C_o}
%\chainedlink{\mathit{cxt}}{\mathit{cxt}}\chainedlink{\underline{C}_i}{\bf \underline{C}_i}\chainedlink{\bf \underline{C}_o}{\underline{C}_o}
%\chainedlink{p_1}{p_1}\chainedlink{p_2}{p_2}\chainedlink{\widetilde{G}_i}{\bf \widetilde{G}_i}\chainedlink{\bf \widetilde{G}_o}{\widetilde{G}_o}\chainend{\silent}
r_1~\overline{G}~r_2~C~\overline{H}~\mathit{cxt}~\underline{C}~\underline{G}~\underline{H}~p_1~p_2~\widetilde{G}
\end{array}
\]
\[
\begin{array}{lcl @{\hspace{0.3cm}}lcl}
\ell_3 & \defeq & 
% \startchain{\silent}\chainedlink{G_i}{\bf G_i}\chainedlink{\bf G_o}{G_o}\chainedlink{r_2}{r_2}\chainedlink{C_i}{\bf C_i}\chainedlink{\bf C_o}{C_o}
%\chainedlink{\mathit{cxt}}{\mathit{cxt}}\chainedlink{\widehat{C}_i}{\bf \widehat{C}_i}\chainedlink{\bf \widehat{C}_o}{\widehat{C}_o}
%\chainedlink{p_1}{p_1}\chainedlink{\widetilde{G}_i}{\bf \widetilde{G}_i}\chainedlink{\bf \widetilde{G}_o}{\widetilde{G}_o}\chainedlink{p_2}{p_2}\chainedlink{\widetilde{G}_i}{\bf \widetilde{G}_i}\chainedlink{\bf \widetilde{G}_o}{\widetilde{G}_o}\chainend{\silent}
r_1~G~\overline{H}~r_2~C~\overline{H}~\mathit{cxt}~\widehat{C}~\underline{G}~\overline{H}~p_1~\widetilde{C}~p_2~\widetilde{G}
\\
\ell'_3 & \defeq & r_1~G~\overline{H}~r_2~C~\overline{H}~\mathit{cxt}~\underline{C}~\underline{G}~\overline{H}~p_1~\widetilde{C}~p_2~\widetilde{G}
% \startchain{\silent}\chainedlink{G_i}{\bf G_i}\chainedlink{\bf G_o}{G_o}\chainedlink{r_2}{r_2}\chainedlink{C_i}{\bf C_i}\chainedlink{\bf C_o}{C_o}
%\chainedlink{\mathit{cxt}}{\mathit{cxt}}\chainedlink{\underline{C}_i}{\bf \underline{C}_i}\chainedlink{\bf \underline{C}_o}{\underline{C}_o}
%\chainedlink{p_1}{p_1}\chainedlink{\widetilde{G}_i}{\bf \widetilde{G}_i}\chainedlink{\bf \widetilde{G}_o}{\widetilde{G}_o}\chainedlink{p_2}{p_2}\chainedlink{\widetilde{G}_i}{\bf \widetilde{G}_i}\chainedlink{\bf \widetilde{G}_o}{\widetilde{G}_o}\chainend{\silent}
\end{array}
\]
}

The labels $\flat_i$ $\flat'_i$, with $i \in \{1,2,3\}$ can be obtained by labels $\ell_i,\ell'_i$ by substituting $G$ with $C$ and viceversa.
\\
Now, it easy to check that $\mathit{Sys}_1$ and $\mathit{Sys}_2$ are bio-similar w.r.t the property $\mathsf{F}$ saying that $G$ and $C$ are simultaneously  produced,
%in $Sys_1$ if and only if $G$ and $C$ are simultaneously  produced in $Sys_2$
formally:
$\mathit{Sys}_1 \sim_{\mathsf{F}} \mathit{Sys}_2$ with $\mathsf{F}= \star :: \widetilde{G} :: \star \wedge  \star :: \widetilde{C} :: \star $.

On the contrary, $\mathit{Sys}_1 \not \sim_{\mathsf{F'}} \mathit{Sys}_2$  with $\mathsf{F'} =  \star :: \widetilde{C} :: \star $, 
because it happens that both  transition labels $\ell_1$  and $\ell'_1$, in $\mathit{Sys}_1$, record the production of $C$, whereas 
transition labels $\flat_1$ and $\flat'_1$ do not.
In fact, the bioHML formula $\mathsf{G}\defeq \langle \mathsf{F'}\rangle{\tt t}$ can be used to distinguish $\mathit{Sys}_1$ from $\mathit{Sys}_2$, as $\mathit{Sys}_1\entails \mathsf{G}$ and $\mathit{Sys}_2\not\entails \mathsf{G}$.

\begin{figure}
\[
\begin{array}{cc}
\xymatrix@R-1pc{
*+[o][F-]{G} \ar@/ ^/[d]^{\ell_1} \ar@/_/[d]^{\ell'_1} \\
*+[o][F-]{C} \ar@/^2pc/[u]^{\ell_2} \ar@/^/[dr]^{\ell'_2} \\ %\ar@(r,u)[]_a\
& *+[o][F-]{GC}\ar@(d,l)^{\ell_3}\ar@(u,r)^{\ell'_3}
}
&
\xymatrix@R-1pc{
*+[o][F-]{C} \ar@/ ^/[d]^{\flat_1} \ar@/_/[d]^{\flat'_1} \\
*+[o][F-]{G} \ar@/^2pc/[u]^{\flat_2} \ar@/^/[dr]^{\flat'_2} \\ 
& *+[o][F-]{GC}\ar@(d,l)^{\flat_3}\ar@(u,r)^{\flat'_3}
}\\
Sys_1 & Sys_2
\end{array}
\]
\caption{The two labelled transition systems of $Sys_1$ and $Sys_2$.}\label{fig:ltss}
\end{figure}