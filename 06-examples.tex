% !TEX root = ./main_RS2L_LNCS.tex
\section{Examples}
\label{ex:examples}
Hereafter, we only take into account transitions whose labels are solid, i.e. they do not contain 
virtual links, and   complete, i.e. thy starts and ends with the $\silent$ symbol.\\

\subsection{Labelled transition system}
This example is inspired by the example in~\cite{BEMR11}, where
a deterministic transition system is coded in the reaction system
framework.
Here we consider the minimal deterministic transition system in Figure~\ref{fig:lts}.
%following the reaction system approach, to code it in the reaction system framework, we have to add
%extra labels, as in Figure~\ref{fig:ndlts2}.
\begin{figure}
\[
\xymatrix{
*+[o][F-]{q} \ar@/ ^/[r]^a \ar@(l,u)[]^b &
*+[o][F-]{w} \ar@/^/[l]^b \ar@(r,u)[]_a
}
\]
\caption{Minimal deterministic labelled transition system.}
\label{fig:lts}
\end{figure}
\noindent
In our chained {\tt link}-calculus encoding, $q$, $w$, $a$, and $b$ became the $entities$, the context can only provide $a$ and $b$ and the reaction rules are as follows:\\
\[
\begin{array}{cl@{\hspace{1cm}} cl}
1 \ &(\{q,a\},\{b\},q) & 2\ &(\{q,b\},\{a\},w)\\
3 \  &(\{w,a\},\{b\},q) & 4\ &(\{w,b\},\{a\},w)\\ 
\end{array}
\]
\paragraph{Encoding of the rules.}
The encoding of the rules for reactions is  given in a parametric way:
\[
\begin{array}{lcl}
P_n(q,b,w,a,q) & \defeq & \startchain{r_n}\chainedlink{q_i}{\noact}\chainedlink{\noact}{q_o}
                                                    \chainedlink{b_i}{\noact}\chainedlink{\noact}{b_o}
                                                     \chainedlink{\overline{w}_i}{\noact}\chainedlink{\noact}{\overline{w}_o}
					        \chainedlink{\overline{a}_i}{\noact}\chainedlink{\noact}{\overline{a}_o}
					        \chainedlink{r_n}{\noact}\chainedlink{\noact}{p_n}
					        \chainedlink{\tilde{q}_i}{\noact}\chainedlink{\noact}{\tilde{q}_o}
			\chainend{p_{n+1}}.P_n(q,b,w,a,q)  \\
			&& +\\
			&&P_n'(\overline{q}) \ +\ P_n'(\overline{b}) \ + \ P_n'(w)\ +\ P_n'(a)\\
			\end{array}
			\]
			\noindent
			where
			\[
\begin{array}{lcl}
P_n'(w) &\defeq &  \startchain{r_n}\chainedlink{w_i}{\noact}\chainedlink{\noact}{w_o}\chainedlink{r_{n+1}}{\noact}\chainedlink{\noact}{p_n} \chainend{p_{n+1}}.P_n'(w) 
% \ + \   \startchain{r_n}\chainedlink{w_i}{\noact}\chainedlink{\noact}{w_o}\chainedlink{r_{n+1}}{\noact}\chainedlink{\noact}{p_n} \chainend{p_{n+1}}.P_n(q,b,w,a,q) \\
%			&&+\\
%			&& \startchain{r_n}\chainedlink{\overline{b}_i}{\noact}\chainedlink{\noact}{\overline{b}_o}\chainedlink{r_{n+1}}{\noact}\chainedlink{\noact}{p_n}\chainend{p_{n+1}}.P_n(q,b,w,a,q) \ + \  \startchain{r_n}\chainedlink{a_i}{\noact}\chainedlink{\noact}{a_o}\chainedlink{r_{n+1}}{\noact}\chainedlink{\noact}{p_n} \chainend{p_{n+1}}.P_n(q,b,w,a,q) 
\end{array}
\]
\noindent
Then we have 
\[
\begin{array}{lcl @{\hspace{0.5cm}}lcl  }
P_1 & \defeq & P_1(q,a,w,b,q) & P_3 & \defeq & P_3(w,a,q,b,w)  \\
P_2 & \defeq & P_2(q,b,w,a,w) &P_4 & \defeq & P_4(w,b,q,a,q) 
\end{array}
\]
\noindent
and we put $r_1 = \silent$, $r_5 = cxt$ and  $p_5 = \silent$.

\paragraph{Encoding of the entities.}
As for reactions, also the encoding of the entities is given in a parametric way.
Here we differentiate the encoding for the entities that are not provided by the context and that can be produced by the reactions, and the ones that can be provided by the context and that are not produced by the reactions.\\
Here, the entities $p$ and $w$ that are not provided by the context:
\[
\begin{array}{lcl@{\hspace{1cm}}lcl}
P (q)&\defeq & \sum_{h=0}^1 (\startchain{q_i}\chainedlink{q_o}{\noact}\chainend{\noact})^h\ \link{\tilde{q}_i}{\tilde{q}_o}.P(q) &
\overline{P} (q)&\defeq & \sum_{h=1}^2 (\startchain{\overline{q}_i}\chainedlink{\overline{q}_o}{\noact}\chainend{\noact})^h\ \link{\tilde{q}_i}{\tilde{q}_o}.P(q)\\
&& + & && +\\
&& \link{q_i}{q_o}.\overline{P}(q) &
&& \sum_{h=1}^2 (\startchain{\overline{q}_i}\chainedlink{\overline{q}_o}{\noact}\chainend{\noact})^h.\overline{P}(q)\\
\end{array}
\]
Then, we have
\[
\begin{array}{lcl @{\hspace{1cm}}lcl}
P_q & \defeq & P(q) & P_w& \defeq & P(w)\\
\end{array}
\]
Here, the entities that are provided by the context:
\[
\begin{array}{lcl @{\hspace{1cm}}lcl}
E(a) & \defeq &  \sum_{h=0}^1 (\startchain{a_i}\chainedlink{a_o}{\noact}\chainend{\noact})^h\ \link{\hat{a}_i}{\hat{a}_o}.E(a) &  \overline{E} (a)&\defeq & \sum_{h=1}^2 (\startchain{\overline{a}_i}\chainedlink{\overline{a}_o}{\noact}\chainend{\noact})^h\ \link{\hat{a}_i}{\hat{a}_o}.E(a)\\
&& + &  && +\\
&& \sum_{h=0}^1 (\startchain{a_i}\chainedlink{a_o}{\noact}\chainend{\noact})^h\ \link{\underline{a}_i}{\underline{a}_o}.\overline{E}(a) &
&& \sum_{h=1}^2 (\startchain{\overline{a}_i}\chainedlink{\overline{a}_o}{\noact}\chainend{\noact})^h\ \link{\underline{a}_i}{\underline{a}_o}.\overline{E}(q)\\
\end{array}
\]
Then, we have
\[
\begin{array}{lcl @{\hspace{1cm}}lcl}
P_a & \defeq & E(a) & P_b& \defeq & E(b)\\
\end{array}
\]
For the context, the encoding follows:
\[
\begin{array}{lcl @{\hspace{0.5cm}}c@{\hspace{0.5cm}} l}
Cxt &\defeq &  \startchain{cxt}\chainedlink{\hat{a}_i}{\noact}\chainedlink{\noact}{\hat{a}_o}\chainedlink{\underline{b}_i}{\noact}\chainedlink{\noact}{\underline{b}_i}\chainend{p_1}.Cxt
&+
&\startchain{cxt}\chainedlink{\hat{b}_i}{\noact}\chainedlink{\noact}{\hat{b}_o}\chainedlink{\underline{a}_i}{\noact}\chainedlink{\noact}{\underline{a}_i}\chainend{p_1}.Cxt\\
\end{array}
\]
We model the context to always offer either  $a$ or $b$, and no both the entities. The reason is that in
the other cases (providing both $a$ and $b$ or neither of them) leads the system to be stuck.\\
Now, we assume that we have an initial configuration containing the entities $q$ and $b$ (with$\tilde{q},\tilde{w},\tilde{a},\tilde{b}$ grouping all the possibile decorations for these names):
\[
\begin{array}{lcl}
Sys  \defeq \ %\restrict{r1,r2,r3,r4,\tilde{q},\tilde{w},\tilde{a},\tilde{b}}
 P_q \mid \overline{P}_w  \mid  P_a  \mid \overline{P}_b  \mid  P_1  \mid  P_2  \mid  P_3  \mid P_4  \mid Cxt.
\end{array}
\] 
Then only the second reaction can be applied, and the label transition follows
{\small
\[
\begin{array}{l}
%\restrict{r1,r2,r3,r4,\tilde{q},\tilde{w},\tilde{a},\tilde{b}}\\
 \startchain{\silent}\chainedlink{\overline{a}_i}{\bf  \overline{a}_i}\chainedlink{\bf \overline{a}_o}{\overline{a}_o}\chainedlink{r_2}{r_2}\chainedlink{q_i}{\bf q_i}\chainedlink{\bf q_i}{q_i}\chainedlink{b_i}{\bf b_i}\chainedlink{\bf b_i}{b_i}\chainedlink{\overline{w}_i}{\bf \overline{w}_i}\chainedlink{\bf \overline{w}_o}{\overline{w}_o}\chainedlink{\overline{a}_i}{\bf \overline{a}_i}\chainedlink{\bf \overline{a}_o}{\overline{a}_o}\chainedlink{r_3}{r_3}\chainedlink{\overline{a}_i}{\bf \overline{a}_i}\chainedlink{\bf \overline{a}_o}{\overline{a}_o}\chainedlink{r_4}{r_4}\chainedlink{\overline{w}_i}{\bf \overline{w}_i}\chainedlink{\bf \overline{w}_o}{\overline{w}_o}\chainedlink{cxt}{cxt}\chainedlink{\hat{a}_i}{\bf \hat{a}_i}\chainedlink{\bf \hat{a}_o}{\hat{a}_o}\chainedlink{\underline{b}_i}{\bf \underline{b}_i}\chainedlink{\bf \underline{b}_o}{\underline{b}_o}\chainedlink{p_1}{p_1}\chainedlink{p_2}{p_2}\chainedlink{\tilde{w}_i}{\bf \tilde{w}_i}\chainedlink{\bf \tilde{w}_o}{\tilde{w}_o}\chainedlink{p_3}{p_3}\chainedlink{p_4}{p_4}\chainend{\silent}
\end{array}
\]}
The parts in bold are provided by the entity processes, the other parts are provided by the processes encoding the reactions and by the process encoding the context (starting at $cxt$ and ended at $p_1$.
In the label we can read that the rules $1$ and $4$ have been not executed as the entity $a$ is absent, 
the rule $3$ has been not applied as the entity $w$ is absent, then only rule $2$ has been applied, and then it has produced entity $w$. Also, in the next state the context will provide entity $a$ and will not entity $b$.

\subsection{A  biological toy example}\label{subsec:toy}


Here we consider a biological toy example where a gene $a$  codes for a protein $T$ when molecules 
$G$ is present and $C$ is absent, and in the opposite situation $a$ codes for protein $T'$.
This behavior is encoded in rules $1$ and $2$.
Then, rule three codes for the production of $C$ when proteins $T$ and $F$ are present, and $T'$ absent;
rule four codes for the production of $G$ when  proteins $T'$ is present and $F$ is absent.
\paragraph{Encoding of the rules.}
The encoding of the rules for reactions is  given in a parametric way:
\[
\begin{array}{lcl}
P_n(a,G,C,T) & \defeq &\startchain{r_n}\chainedlink{a_i}{\noact}\chainedlink{\noact}{a_o}
                                                    \chainedlink{G_i}{\noact}\chainedlink{\noact}{G_o}
                                                     \chainedlink{\overline{C}_i}{\noact}\chainedlink{\noact}{\overline{C}_o}
					        \chainedlink{r_{n+1}}{\noact}\chainedlink{\noact}{p_n}
					        \chainedlink{\tilde{T}_i}{\noact}\chainedlink{\noact}{\tilde{T}_o}
			\chainend{p_{n+1}}.P_n(a,G,C,T)  \\
				&& +\\
			&&P_n'(\overline{a}) \ +\ P_n'(\overline{G}) \ + \ P_n'(C)\\
			\end{array}
			\]
			\noindent
			where
			\[
\begin{array}{lcl}
P_n'(C) &\defeq &  \startchain{r_n}\chainedlink{C_i}{\noact}\chainedlink{\noact}{C_o}\chainedlink{r_{n+1}}{\noact}\chainedlink{\noact}{p_n} \chainend{p_{n+1}}.P_n'(C) 
% \ + \   \startchain{r_n}\chainedlink{w_i}{\noact}\chainedlink{\noact}{w_o}\chainedlink{r_{n+1}}{\noact}\chainedlink{\noact}{p_n} \chainend{p_{n+1}}.P_n(q,b,w,a,q) \\
%			&&+\\
%			&& \startchain{r_n}\chainedlink{\overline{b}_i}{\noact}\chainedlink{\noact}{\overline{b}_o}\chainedlink{r_{n+1}}{\noact}\chainedlink{\noact}{p_n}\chainend{p_{n+1}}.P_n(q,b,w,a,q) \ + \  \startchain{r_n}\chainedlink{a_i}{\noact}\chainedlink{\noact}{a_o}\chainedlink{r_{n+1}}{\noact}\chainedlink{\noact}{p_n} \chainend{p_{n+1}}.P_n(q,b,w,a,q) 
\end{array}
\]

Then we have 
\[
\begin{array}{lcl @{\hspace{0.5cm}}lcl @{\hspace{0.5cm}}lcl }
P_1 & \defeq & P_1(a,G,C,T) &   P_2 & \defeq & P_2(a,C,G,T') &   P_3 & \defeq & P_3(F,T,T',C)  \\
\end{array}
\]
\[
\begin{array}{lcl}
P_4 & \defeq &\startchain{r_4}\chainedlink{T'_i}{\noact}\chainedlink{\noact}{T'_o}
                                                     \chainedlink{\overline{F}_i}{\noact}\chainedlink{\noact}{\overline{F}_o}
					        \chainedlink{cxt}{\noact}\chainedlink{\noact}{p_4}
					        \chainedlink{\tilde{G}_i}{\noact}\chainedlink{\noact}{\tilde{G}_o}
			\chainend{\silent}.P_4  \\
				&& +\\
&&\startchain{r_4}\chainedlink{\overline{T'}_i}{\noact}\chainedlink{\noact}{\overline{T'}_o}\chainedlink{cxt}{\noact}\chainedlink{\noact}{p_4} \chainend{\silent}.P_4 \ +\  \startchain{r_4}\chainedlink{F_i}{\noact}\chainedlink{\noact}{F_o}\chainedlink{cxt}{\noact}\chainedlink{\noact}{p_4} \chainend{\silent}.P_4
\end{array}
\]
\noindent
and we put $r_1 = \silent$.
\paragraph{Encoding of the entities.}
As for reactions, also the encoding of the entities is given in a parametric way.
Here we differentiate three types of encodings: (1) for the entities that are not  provided by the context and  can be produced by the reactions; (2) for the entities that can be provided by the context and can be  produced by the reactions; (3) for the entities that are only provided by the context.\\
Here, the entities $T$ and $T'$ that can be produced by the reactions and that are not provided by the context:
\[
\begin{array}{lcllcl}
P (T,\tilde{T})&\defeq & \sum_{h=0}^1 (\startchain{T_i}\chainedlink{T_o}{\noact}\chainend{\noact})^h\ \link{\tilde{T}_i}{\tilde{T}_o}.P (T,\tilde{T})
& +  && \link{T_i}{T_o}.P (\overline{T},\tilde{T}) \\
%\overline{P} (q)&\defeq &  \sum_{h=0}^1 (\startchain{\overline{T}_i}\chainedlink{\overline{T}_o}{\noact}\chainend{\noact})^h\link{\tilde{T}_i}{\tilde{T}_o}.P(q)\\
%&& + & && +\\
%&& \link{T_i}{T_o}.\overline{P}(T) &
%&& \link{\overline{T}_i}{\overline{T}_o}.\overline{P}(T)\\
\end{array}
\]
Then, we have
\[
\begin{array}{lcl @{\hspace{0.5cm}}lcl @{\hspace{0.5cm}}lcl @{\hspace{0.5cm}}lcl}
P_T & \defeq & P(T,\tilde{T})& \overline{P}_T & \defeq & P(\overline{T},\tilde{T}) & P_{T'}& \defeq & P(T',\tilde{T'}) & \overline{P}_{T',\tilde{T'}}& \defeq & P(\overline{T'},\tilde{T'})\\\
\end{array}
\]
Here, the entities that can be produced by the reactions and that can be  provided by the context:
\[
\begin{array}{lcl @{\hspace{1cm}}lcl}
P(C,\hat{C},\underline{C},\tilde{C}) & \defeq &  \sum_{h=0}^1 (\startchain{C_i}\chainedlink{C_o}{\noact}\chainend{\noact})^h\ \link{\hat{C}_i}{\hat{C}_o} (\startchain{\noact}\chainedlink{\noact}{\tilde{C}_i}\chainend{\tilde{C}_o})^h.   P(C,\hat{C},\underline{C},\tilde{C})\\
&&+\\
&& \sum_{h=0}^1 (\startchain{C_i}\chainedlink{C_o}{\noact}\chainend{\noact})^h\ \link{\underline{C}_i}{\underline{C}_o}.P(\overline{C},\hat{C},\underline{C},\tilde{C})\\
&&+\\
 && \sum_{h=0}^1 (\startchain{C_i}\chainedlink{C_o}{\noact}\chainend{\noact})^h\ \link{\underline{C}_i}{\underline{C}_o} \startchain{\noact}\chainedlink{\noact}{\hat{C}_i}\chainend{\hat{C}_o}.   P(C,\hat{C},\underline{C},\tilde{C}) \\
 %&  \overline{E} (a)&\defeq & \sum_{h=1}^2 (\startchain{\overline{a}_i}\chainedlink{\overline{a}_o}{\noact}\chainend{\noact})^h\ \link{\hat{a}_i}{\hat{a}_o}.E(a)\\
%&&  &  && +\\
%&& 
%&
%&& \sum_{h=1}^2 (\startchain{\overline{a}_i}\chainedlink{\overline{a}_o}{\noact}\chainend{\noact})^h\ \link{\underline{a}_i}{\underline{a}_o}.\overline{E}(q)\\
\end{array}
\]
Then, we have
\[
\begin{array}{lcl @{\hspace{1cm}}lcl@{\hspace{0.4cm}}lcl@{\hspace{0.4cm}}lcl}
P_C & \defeq &  P(C,\hat{C},\underline{C},\tilde{C})&  P_G& \defeq &   P(G,\hat{G},\underline{G},\tilde{G}) \\
\overline{P}_C & \defeq &  P(\overline{C},\hat{C},\underline{C},\tilde{C}) & 
\overline{P}_G& \defeq &   P(\overline{G},\hat{G},\underline{G},\tilde{G})\\
\end{array}
\]

Here, the encoding of the entity $F$ that can be provided by the context:
\[
\begin{array}{lcllcl}
P_F&\defeq & \sum_{h=0}^1 (\startchain{F_i}\chainedlink{F_o}{\noact}\chainend{\noact})^h\ \link{\hat{F}_i}{\hat{F}_o}.P_F
& +  && \sum_{h=0}^1 (\startchain{F_i}\chainedlink{F_o}{\noact}\chainend{\noact})^h. \link{\underline{F}_i}{\underline{F}_o}.\overline{P}_F \\
\overline{P}_F&\defeq & \sum_{h=0}^1 (\startchain{\overline{F}_i}\chainedlink{\overline{F}_o}{\noact}\chainend{\noact})^h\ \link{\hat{F}_i}{\hat{F}_o}.P_F
& +  && \sum_{h=0}^1 (\startchain{\overline{F}_i}\chainedlink{\overline{F}_o}{\noact}\chainend{\noact})^h. \link{\underline{F}_i}{\underline{F}_o}.\overline{P}_F \\
%\overline{P} (q)&\defeq &  \sum_{h=0}^1 (\startchain{\overline{T}_i}\chainedlink{\overline{T}_o}{\noact}\chainend{\noact})^h\link{\tilde{T}_i}{\tilde{T}_o}.P(q)\\
%&& + & && +\\
%&& \link{T_i}{T_o}.\overline{P}(T) &
%&& \link{\overline{T}_i}{\overline{T}_o}.\overline{P}(T)\\
\end{array}
\]
Now, the context follow can non deterministically provide the entities: $C$, $G$ $F$:\\
\[
\begin{array}{lcl}
 Cxt & \defeq &
 \sum_{\tiny \begin{array}{l}C^{\star} \in \{C,\overline{C}\}\\G^{\star} \in \{G,\overline{G}\}\\
 F^{\star} \in \{F,\overline{F}\}
  \end{array}}
 \startchain{cxt} \chainedlink{C_i^{\star}}{\noact}\chainedlink{\noact}{C_o^{\star}}\chainedlink{G_i^{\star}}{\noact}\chainedlink{\noact}{G_o^{\star}}
 \chainedlink{F_i^{\star}}{\noact}\chainedlink{\noact}{F_o^{\star}}\chainend{p_1}.Cxt\\
\end{array}
\]

Now, to show a possible composition of a transition label, we assume an system where the entities $a$, $T'$, and $G$ are present:
\[
\begin{array}{lcl}
Sys &\defeq &%\restrict{\tilde{C},\tilde{G},\tilde{F}, \tilde{T},\tilde{T'}}(
P_a\mid \overline{P}_C\mid P_G\mid \overline{P}_F \mid \overline{P}_T \mid P_{T'} \mid P_1 \mid P_2 \mid P_3 \mid P_4
\end{array}
\]
In the above configuration, reactions $1$ and $3$ can be applied, and also we assume that the context will provide the entity $F$, that will be available in the target configuration.
The transition label appears as:
\[
\tiny
\begin{array}{l}
%\restrict{r1,r2,r3,r4,a,\tilde{C},\tilde{G},\tilde{F},\tilde{T},\tilde{T'}}\\
 \startchain{\silent}\chainedlink{a_i}{\bf  a_i}\chainedlink{\bf a_o}{a_o}
 \chainedlink{G_i}{\bf  G_i}\chainedlink{\bf G_o}{G_o}
 \chainedlink{\overline{C}_i}{\bf \overline{C}_i}\chainedlink{\bf \overline{C}_o}{\overline{C}_o}
 \chainedlink{r_2}{r_2}\chainedlink{G_i}{\bf G_i}\chainedlink{\bf G_o}{G_o}
\chainedlink{r_3}{r_3}\chainedlink{T'_i}{\bf T'_i}\chainedlink{\bf T'_o}{T'_o}\chainedlink{r_4}{r_4}\chainedlink{T'_i}{\bf T'_i}\chainedlink{\bf T'_o}{T'_o}\chainedlink{\overline{F}_i}{\bf \overline{F}_i}\chainedlink{\bf \overline{F}_o}{\overline{F}_o}\chainedlink{cxt}{cxt}
\chainedlink{\underline{C}_i}{\bf \underline{C}_i}\chainedlink{\bf \underline{C}_o}{\underline{C}_o}
\chainedlink{\underline{G}_i}{\bf \underline{G}_i}\chainedlink{\bf \underline{G}_o}{\underline{G}_o}
\chainedlink{\hat{F}_i}{\bf \hat{F}_i}\chainedlink{\bf \hat{F}_o}{\hat{F}_o}\chainedlink{p_1}{p_1} 
 \chainedlink{\tilde{T}_i}{\bf \tilde{T}_i}\chainedlink{\bf \tilde{T}_o}{\tilde{T}_o}
  \chainedlink{p_2}{p_2}\chainedlink{p_3}{p_3}\chainedlink{p_4}{p_4} \chainedlink{\tilde{G}_i}{\bf \tilde{G}_i}\chainedlink{\bf \tilde{G}_o}{\tilde{G}_o} \chainend{\silent}
\end{array}
\]



