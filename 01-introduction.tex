% !TEX root = ./main_RS2L_LNCS.tex

\section{Introduction}

Natural Computing is an emerging area of research which has two main 
aspects: human designed computing inspired by nature, and computation 
performed in nature. Reaction Systems (RSs)~\cite{BEMR11} are 
% model of
% formal computation 
a rewriting formalism
%model 
inspired by the way biochemical reactions take place in living 
cells.  
This theory has already shown to be relevant in several different 
fields, such as computer science~\cite{MPR15}, 
biology~\cite{ABP14,CMMBM12,Az17,BarbutiGLM16}, 
molecular chemistry~\cite{OY16}.
%,
% and it
%is continuously developing, with a growing number of workshops and
%international meetings. 
Reaction Systems formalise the mechanisms of biochemical systems, 
such as {\em facilitation} and {\em inhibition}. 
As a qualitative approximation of the real biochemical reactions, they
% do not take
%into account the quantity of reagents effectively present. They
consider if a necessary reagent is or not present, and likewise they
consider if an inhibiting molecule is or not present. 
The  possible reactants and inhibitors are called `entities'.
RSs model in a direct way the interaction of a living cell
with the environment (called `context'). However, two RSs are seen
as independent models and do not interact.


%Formally, Reaction Systems are defined as a rewriting formalism.
In this paper, we present an encoding from RSs, 
to the open multiparty process algebra c\CNA,\footnote{After `chained Core Network 
Algebra'.} a variant of the {\tt link}-calculus~\cite{BodeiBB12,BBB17} without name mobility.
This formalism allows several processes to synchronise and 
communicate altogether, at the same time, with a new interaction
mechanism based on links and link chains. 
The initial motivation for introducing this mechanism was to encode
Mobile Ambients~\cite{CardelliG00}, getting a much stronger operational
correspondence than any available in the literature, such as the one in~\cite{B16}.
Later it was shown that the {\tt link}-calculus allowed one to easily encode calculi for biology equipped with membranes, as in~\cite{BodeiBBC14}.

We illustrate our embedding by means of some examples coming
both from the computer science and the biological field.
%and then we consider a more complex example, by modeling a RS 
%representing a regulatory network for \emph{lac} operon, presented in~\cite{CMMBM12}. 
We also show that our embedding preserves the main
features of RSs, and prove its correctness and completeness
from the operational semantics viewpoint.

Then, we present a methodology for verifying formally
properties of Reaction Systems. The classical notion of bisimulation for 
process algebras allows to consider two processes as equivalent 
when one process can simulate all the actions executed 
by the other one and vice versa. In this paper we define a new
notion of bisimilarity which takes into account the characteristics
of biological systems. We define a new {\em bio-simulation} having in mind
the possibility that two interacting systems may be compared w.r.t.
a subset of the possible biological actions. This is useful for concentrating 
on the sub-model that one may need to consider for a specific study or application,
without getting lost in the complexity of the full biological system, or network.
The notion of bio-simulation relies on a new and simple assertion language,
which allows to focus on some properties of the Reaction Systems to be verified.
In fact, bio-similarity is parametric on a given assertion of interest.
Then we prove that the well-known logical characterisation of bisimilarity in terms of Hennessy-Milner Logic~\cite{HM80} (HML)
can be extended to the case of bio-similarity by tailoring HML formulas to
the same assertion of interest used in bio-similarity.  As HML is a powerful
formalism for specifying properties of labeled transition systems, 
its variant introduced here looks a suitable option to specify formulas 
over complex labels by abstracting from unnecessary details.

Our main contributions are as follows:
\begin{itemize}
\item the behaviour of the context, for each single 
entity,
%\footnote{Entities that belong to the `domain' of a RS.}
can be
specified in a recursive way, as ordinary processes;
\item along the same lines, non-deterministic contexts can be seamlessly expressed in our setting; 
\item when deterministic contexts are considered, the c\CNA \
computation is deterministic and mimics the evolution of the corresponding RS;
\item we define an assertion language to specify local properties of 
Reaction Systems and exploit it to define a novel notion of behavioural equivalence for Reaction Systems, called bio-similarity;
\item we show that our assertions can be used to extend the Hennessy-Milner logic to our framework, in such a way that logical equivalence of processes coincide with bio-similarity;
\item we sketch how one can express the behaviour of entity mutation,
%change the behaviour of  each single entity $s$  in a
%way that a mutation $s'$ can be expressed; 
in such a way that the mutated entity
$s'$ can  take part to only a subset of rules requiring entity $s$;
\item we show that with a little coding effort, our encoding allows two RSs to
communicate; i.e. we can model scenarios where a subset of those entities that the context  can
provide, are instead provided by a second RS.
\end{itemize}

The main drawback of our proposal, is that the c\CNA \ translation is
verbose. Nevertheless it is clear that our
translation can be automatised by means of a proper front-end in
an implementation of the {\tt link}-calculus. 
The examples that we propose also show that some optimisations are possible
to reduce the coding when suitable assumptions are made about the provision of entities.

As we have remarked, in our translation, Reaction Systems 
get the ability to interact between them in a synchronized manner. 
This interaction is not foreseen in
the basic RS framework, as it can only happen with the context.
By exploiting recursion, the kind of interactions which can be 
defined can be complex and expressive.
Example~\ref{ex:twoRS} and more in general the discussion in 
Section~\ref{sec:discussion} show that 
the interaction between RSs can help to model new scenarios.

\paragraph{Related work}

Process calculi have been used successfully to model
biological processes, see~\cite{BBDFH18} for a recent survey.
We are not aware of any SOS operational semantics for the RS; it seems not the case
that a deterministic behaviour, as the one of the RS, once the initial state is 
fixed, could be defined by inference rules that classically are applied 
to define non-deterministic transition systems.
The reversible computation paradigm for RS extends the RS framework  by
allowing  backward computations as well as forward computations; for this 
goal a set of inference rules for rewriting logic has been 
defined in~\cite{10.1007/978-3-319-73359-3_3}
to exactly keep trace of those elements that  dissolve in the next 
computation step.


A preliminary version of this paper appeared in~\cite{BBF19}.
There are several major differences w.r.t. the paper~\cite{BBF19}, which is here extended as follows:
\begin{itemize}
\item we present several new examples to illustrate our framework,
and give more detailed explanations about its use;
\item we introduce an assertion language to specify the properties of Reaction Systems;
\item we define the notion of bio-simulation to relate Reaction Systems and compare their behaviours;
\item we show that our assertion language allows to specify properties extending to our
framework the Hennessy-Milner logic;
\item we include here all proofs of main results.
\end{itemize}
 
\paragraph{Structure of the paper} 

Section \ref{sec:rs} 
describes RSs and their semantics (interactive processes).
Section \ref{sec:ccna} describes briefly the c\CNA \ process algebra and
its operational semantics.
Section \ref{sec:trans} defines the embedding of RSs in
c\CNA \ processes and shows some simple examples to illustrate 
it.
Section~\ref{ex:examples} presents a couple of examples with computer science and biological applications.
%Section \ref{ex:lactose} shows a more complex example taken from
%the literature on RSs and illustrating a \emph{lac} operon.
Section~\ref{sec:biosimulation} presents a methodology for the formal verification of
properties of Reaction Systems that are expressed in a novel assertion language.
Section~\ref{sec:discussion} presents some features and 
advantages of our embedding for the compositionality of RSs.
Finally, Section~\ref{sec:conclusion}
discusses future work, and concludes.


