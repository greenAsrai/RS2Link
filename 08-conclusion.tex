% !TEX root = ./main_RS2L_LNCS.tex

\section{Conclusion}\label{sec:conclusion}

In this paper we have introduced a variant of the {\tt link}-calculus
where prefixes are link chains and no more single links, as it was 
briefly described in the future work section in~\cite{BBB17}.
This variant allowed us to define 
%shown 
a faithful embedding of Reaction Systems,
an emerging formalism to model computationally biochemical systems.
%into multilink processes.  
This translation shows several benefits.
For instance, in our paper the context works non deterministically and it 
is recursively defined.
%  the context behaviour is expressed recursively and no be expressed recursively. 
Also, entity mutations can be expressed easily and different reaction systems can
communicate between them.

We have then modified the classical notion (in concurrency) of bisimulation
making it more suitable for proving properties of Reaction Systems in our
framework.
We have defined a new assertion language, which allows us to specify
the properties to be verified, and have shown that it extends Hennessy-Milner logic
to our framework.
We believe that our work can make 
possible to investigate how to integrate our methodology
with other formal techniques to prove 
properties of the modeled systems \cite{CFHOT15,OCHF16,BBGLBH2017}.

We believe that our embedding can contribute to extend the applications
of Reaction Systems to diverse fields of computer science, and life
sciences.
As we have already mentioned, the evolution of each process resulting from our embedding 
is deterministic, thus we do not have the problem of having infinitely many transitions
in the produced labelled transition system. In any case, we can exploit the implementation of  the symbolic semantics of the {\tt link}-calculus~\cite{BrodoO17} that can be found in~\cite{tool}.


As future work, we plan to implement a prototype of our framework,
with an automatic translation from RSs to  {\tt link}-calculus. 
We believe that our work can also help to extend the framework
of RSs towards a model which can improve the communication
between different RSs. 
