% !TEX root = ./main_RS2L_LNCS.tex

\section{Introduction}

Natural Computing is an emerging area of research which has two main 
aspects: human designed computing inspired by nature, and computation 
performed in nature. Reaction Systems (RSs)~\cite{BEMR11} are a 
% model of
% formal computation 
rewriting formalism
%model 
inspired by the way biochemical reactions take place in living 
cells.  
This theory has already shown to be relevant in several different 
fields, such as computer science~\cite{MPR15}, 
biology~\cite{ABP14,CMMBM12,Az17,BarbutiGLM16}, 
molecular chemistry~\cite{OY16}.
%,
% and it
%is continuously developing, with a growing number of workshops and
%international meetings. 
Reaction Systems formalise the mechanisms of biochemical systems, 
such as {\em facilitation} and {\em inhibition}. 
As a qualitative approximation of the real biochemical reactions, they
% do not take
%into account the quantity of reagents effectively present. They
consider if a necessary reagent is or not present, and likewise they
consider if an inhibiting molecule is or not present. 
The  possible reactants and inhibitors are called `entities'.
RSs model in a direct way the interaction of a living cell
with the environment (called `context'). However, two RSs are seen
as independent models and do not interact.


%Formally, Reaction Systems are defined as a rewriting formalism.
In this paper, we present an encoding from RSs, 
to the open multiparty process algebra c\CNA,\footnote{After `chained Core Network 
Algebra'.} a variant of the {\tt link}-calculus~\cite{BodeiBB12,BBB17} without name mobility.
This formalism allows several processes to synchronise and 
communicate altogether, at the same time, with a new communicating
mechanism based on links and link chains. 
Our initial motivation for introducing this mechanism was to encode
Mobile Ambients~\cite{CardelliG00}, getting a much stronger operational
correspondence than any available in the literature, such as the one in~\cite{B16}.
 This allowed us to easily encode calculi for biology equipped with membranes, as in~\cite{BodeiBBC14}.
Process calculi have been used successfully to model
biological processes, see~\cite{BBDFH18} for a recent survey.
We illustrate our embedding by means of some simple basic examples,
and then we consider a more complex example, by modeling a RS 
representing a regulatory network for \emph{lac} operon, presented in~\cite{CMMBM12}. We also show that our embedding preserves the main
features of RSs, and prove its correctness and completeness.
Our main contributions are as follows:
\begin{itemize}
\item the behaviour of the context, for each single 
entity,
%\footnote{Entities that belong to the `domain' of a RS.}
can be
specified in a recursive way as an ordinary process;
\item we can express the behaviour of entity mutation,
%change the behaviour of  each single entity $s$  in a
%way that a mutation $s'$ can be expressed; 
in such a way that the mutated entity
$s'$ can  take part to only a subset of rules requiring entity $s$;
\item with a little coding effort, two RSs can
communicate; i.e.  a subset of those entities that the context  can
provide, are then provided by a second RS;
\item as our translation results in  a  c\CNA \ system, from
each state only one transition can be generated, thus the c\CNA \
computation is fully deterministic.
\end{itemize}

The main drawback of our proposal, is that the c\CNA \ translation is
verbose. Nevertheless it is clear that our
translation can be automatised by means of a proper front-end in
an implementation of the {\tt link}-calculus.

As we have remarked, in our translation, Reaction Systems 
get the ability to interact between them in a synchronized manner. 
This interaction is not foreseen in
the basic RS framework, as it can only happen with the context.
By exploiting recursion, the kind of interactions which can be 
defined can be complex and expressive.
Example~\ref{ex:twoRS} and more in general the discussion in 
Section~\ref{sec:discussion} show that 
the interaction between RSs can help to model new scenarios.
\paragraph{Related work}
We are not aware of any inference rules for the RS; it seems not the case
that a deterministic behavior, as the one of the RS, once the initial state is 
fixed, could be defined by inference rules that classically are applied 
to define non-deterministic transition systems.
The reversible computation paradigm for RS extends the RS framework  by
allowing  backward computations as well as forward computations; for this 
goal a set of inference rules for rewriting logic has been 
defined in~\cite{10.1007/978-3-319-73359-3_3}
to exactly keep trace of those elements that  dissolve in the next 
computation step.
 
\paragraph{Structure of the paper.} Section \ref{sec:rs} 
describes RSs and their semantics (interactive processes).
Section \ref{sec:ccna} describes briefly the c\CNA \ process algebra and
its operational semantics.
Section \ref{sec:trans} defines the embedding of RSs in
c\CNA \ processes and shows some simple examples to illustrate 
it.
Section \ref{ex:lactose} shows a more complex example taken from
the literature on RSs and illustrating a \emph{lac} operon.
Section~\ref{sec:discussion} presents some features and 
advantages of our embedding for the compositionality of RSs.
Finally, Section~\ref{sec:conclusion}
discusses future work, and concludes.


