% !TEX root = ./main_TCS.tex

\section{An overview of Reaction Systems}
\label{sec:rs}

Natural Computing is concerned with human-designed computing inspired by 
nature as well as with computation taking place in nature.
The theory of Reaction Systems~\cite{BEMR11} 
was born in the field of Natural Computing 
to model the behaviour of biochemical reactions taking place in living cells. 
Despite its initial aim, this formalism has shown to be quite useful 
not only for modeling biological phenomena, but also for
the contributions which is giving to computer science~\cite{MPR15}, 
theory of computing, 
mathematics, biology~\cite{ABP14,CMMBM12,Az17,BarbutiGLM16}, 
and molecular chemistry~\cite{OY16}.
%
Here we briefly review the basic notions of RSs, see~\cite{BEMR11} for more details.

The mechanisms that are at the basis of biochemical reactions and thus 
regulate the functioning of a living cell, are 
{\em facilitation} and {\em inhibition}. These mechanisms are 
reflected in the basic definitions of Reaction Systems.

\begin{definition}[Reaction]
A reaction is a triplet $a = (R,I,P)$, where $R, I, P$ are finite,  
non empty sets  and
$R \cap I = \emptyset$. If $S$ is a set such that  $R, I, P \subseteq S$, 
then $a$ is a reaction in $S$.
\end{definition}

{\color{red} The set $S$ is called the entity set}, the sets 
$R, I, P$ are also written $R_a, I_a, P_a$ and called the 
\emph{reactant set} of $a$, the  
\emph{inhibitor set} of $a$, and the \emph{product set} of $a$, respectively. 
All reactants are needed for the reaction to take place.
Any inhibitor blocks the reaction if it is present. Products are the outcome of the reaction.
Also,  $R_a \cup I_a$ is the set of the resources of $a$ and $ rac(S)$
denotes the set of all reactions in $S$.
%
Because $R$ and $I$ are non empty, all products are produced from at least one reactant and every reaction can be inhibited in some way. 
Sometimes artificial inhibitors are used that are never produced by any reaction.
For the sake of simplicity, in some examples, we will allow $I$ to be empty.
%Note that since $R_a$ and $I_a$ are both nonempty, we always 
%have $|M_a| \geq 2$.

\begin{definition}[Reaction System]
A Reaction System (RS) is an ordered pair ${\cal A} = (S, A)$ 
such that $S$ is a finite sete {\color{red} of  entities, and $A \subseteq rac(S)$ is a set over $S$}.
%The set $S$ is called the background (set) of $A$.\\
%The main notion concerning the functioning of a reaction system is defined 
%as follows.
\end{definition}

The set $S$ is called the \emph{background set} of ${\cal A}$, its elements are called \emph{entities},
they represent molecular substances (e.g., atoms, ions, molecules) that may be present
in the states of a biochemical system.
% modeled by $A$. 
The set $A$ is the set of \emph{reactions} of $\mathcal{A}$.
Since $S$ is finite, so is $A$: 
we denote by $|A|$ the number of reactions in $A$.

\begin{definition}[Reaction Result]
Let $W$ be a finite subset of  entities $S$.
\begin{enumerate}
\item Let $a$ be a reaction in $S$. Then $a$ is enabled by the entities in the set $W$, denoted by $\mathit{en}_a(W)$, 
if $R_a \subseteq W$, i.e. all the reactants of $a$ are in $W$,
and $I_a \cap W = \emptyset$, i.e. none of the inhibitors of $a$ are in $W$. The result of $a$ on $W$, denoted by $\mathit{res}_a(W)$, 
is defined by:
$\mathit{res}_a(W) =P_a$, if $\mathit{en}_a(W)$, and $\mathit{res}_a(W) = \emptyset$ otherwise.
\item Let $A$ be a finite set of reactions. The result of $A$ on $W$, denoted 
by $\mathit{res}_A(W)$,  is defined
by: $\mathit{res}_A(W) = \bigcup_{a \in A} \mathit{res}_a(W)$.
\end{enumerate}
\end{definition}


%\medskip
%\noindent
%{\bf Important Assumptions}\\
The theory of Reaction Systems is based on the following assumptions.
\begin{itemize}
\item {\bf No permanency.} An entity of a set $W$ vanishes unless it is sustained 
by a reaction. This reflects the fact that a living cell would die for lack 
of energy, without chemical reactions.
\item {\bf No counting.} The basic model of RSs is very abstract and 
qualitative, i.e. the quantity of entities that are present 
in a cell is not taken into account. 
%It is just analyzed the behavior of 
%the system based on the reactions that take place.
\item {\bf Threshold nature of resources.} From the 
previous item,
we assume that either an entity is available and there is enough of it 
(i.e. there are no conflicts), or it is not available at all.
\end{itemize}




The dynamic behaviour of a RS is formalized in terms of 
{\em interactive processes}.


\begin{definition}[Interactive Process]
Let ${\cal A} = (S, A)$ be a RS and let $n$
% \geq 0$.
be a nonnegative integer;   
An $n$-step
\emph{interactive process} in ${\cal A}$ is a pair $\pi = (\gamma, \delta)$ of 
finite sequences s.t.
%
%\begin{center}
$ \gamma=\{C_i\}_{i\in[0,n]}$  and $\delta=\{D_i\}_{i\in[0,n]} $
%$\gamma = C_0, \dots , C_n$ and $\delta = D_0, \dots , D_n$, 
%\end{center}
%\noindent
where 
$ C_{i}, D_{i}  \subseteq S$ for any $i\in[0,n]$, $D_0 = \emptyset$, and 
%$C_0,\dots, C_n, D_0,\dots, D_n  \subseteq S$, $D_0 = \emptyset$, and 
$D_i =  \mathit{res}_{A}(D_{i-1} \cup C_{i-1})$ for any $i \in [1,n]$.
% $i \in \{1,\dots, n\}$.
\end{definition}

Living cells are seen as open systems that continuously react with 
the external environment, in discrete steps. 
The sequence $\gamma$ is the {\em context sequence} of $\pi$ and 
represents the influence of the environment on the 
RS.
%Reaction System.
The sequence $\delta$ is
the {\em result sequence} of $\pi$ and it is entirely determined by $\gamma$ and $A$.
%
The sequence $\tau = W_{0}, \ldots, W_{n}$ with $W_{i} = C_{i} \cup D_{i}$, 
for any $i \in [0, n]$
%, and $D_{0} = \emptyset$ 
is called a \emph{state sequence}.
%
Each state $W_{i}$ in a state sequence
is the union of two sets: the context $C_{i}$
at step $i$ and the result $D_i=\mathit{res}_{A}(W_{i-1})$ from the previous step.

%\begin{definition}[State Sequence]
%\label{def:statesequence}
% A {\em state sequence} $\tau$ of an interactive process $\pi$:
%
%\begin{center}
%
%\end{center}
%\end{definition}

%\begin{definition} 
%A process $\pi$, or a state 
%sequence $\tau = W_{0}, \ldots, W_{n}$, is said to be 
%{\em context-independent} when $C_{i} 
%\subseteq D_{i}$, $\forall i \in \{i, \ldots, n \}$.
  %\end{definition}

For technical reasons, we extend the notion of an interactive process to deal with 
infinite sequences.

\begin{definition}[Extended Interactive Process]
Let ${\cal A}=(S,A)$ be a RS, and let $\pi =(\gamma,\delta)$  be an  $n$-step interactive process,
with $ \gamma=\{C_i\}_{i\in[0,n]}$ and  $\delta=\{D_i\}_{i\in[0,n]}$
%, $\delta= D_0, \dots D_n$.
Then, we let $\pi^\infty= (\gamma^\infty,\delta^\infty)$ be the extended interactive process of  $\pi$,
defined as $  \gamma^\infty=\{C_i'\}_{i\in\mathbb{N}}$, $\delta^\infty=\{D_i'\}_{i\in\mathbb{N}}$, where:
$$
C'_j = \left\{
\begin{array}{ll}
C_j & \mbox{if $j\in[0,n]$}\\
\emptyset & \mbox{if $j>n$}
\end{array}
\right.
\qquad
D'_j = \left\{
\begin{array}{ll}
D_0 & \mbox{if $j=0$}\\
\mathit{res}_A(D_{j-1}' \cup C_{j-1}') & \mbox{if ($j\geq 1$)}
\end{array}
\right.
$$
% $C_j' = C_j$ for $j\in[0,n]$ and $C_j' =\emptyset$ for $j>n$,   $D_0' = D_0$ and  $D_j' = \mathit{res}_A(D_{j-1}' \cup C_{j-1}')$ for $j\geq 1$.
\end{definition}

Given an extended interactive process $\pi =(\gamma,\delta)$, we denote by $\pi^k$ the shift of $\pi$ starting at the $k$-th state sequence; formally we let $\pi^k=(\gamma^k,\delta^k)$ with
$\gamma^k=\{C_i'\}_{i\in\mathbb{N}}$, $\delta^k=\{D_i'\}_{i\in\mathbb{N}}$ with 
  $C_0'=C_k \cup D_k$, $D'_0 = \emptyset$, and   $C_i'=C_{i+k}$, $D_i'= D_{i+k}$ for any $i\geq 1$.





