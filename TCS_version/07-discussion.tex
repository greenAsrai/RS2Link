% !TEX root = ./main_TCS.tex
%\section{Discussion}

\section{Towards Enhanced  Reaction Systems}
\label{sec:discussion}

Our encoding increases the expressivity of  RS  concerning:
%the behaviour of the context,
the possibility of alternative behaviour of
mutated entities, and  the communication between two different 
RSs.
%Reaction Systems.
It is important to note that, when the context is deterministic, our encoding guarantees that from each state, 
in the  c\CNA \ transition system, only one state is reachable, 
as the dynamics is totally deterministic.

%\subsection{Recursive contexts}
%In RS, the behaviour of the context is finite. For the first $n$ steps, it is specified 
% which are the entities that are provided from the context. 
%Using c\CNA\ we can describe in a natural way the behaviour of the context in a recursive way. 
%Then, the context behaviour would not necessarily end after $n$ steps,
%and  could be infinite.
%For example, in an extended interactive process, we may want that the entity $s$ is 
%intermittently provided by the context every two steps: 
%\[
%\begin{array}{lcll}
%Cxt_s & \defeq &  \startchain{cxt_j}\chainedlink{\hat{s}_i}{\noact}\chainedlink{\noact}{\hat{s}_o}\chainend{cxt_{j+1}}.Cxt_{s'}, &\mbox{ context provides $s$;} \\
%Cxt_{s'} & \defeq & \startchain{cxt_j}\chainedlink{\underline{s}_i}{\noact}\chainedlink{\noact}{\underline{s}_o}\chainend{cxt_{j+1}}.Cxt_{s''},& \mbox{ context doesn't provide $s$;} \\
%Cxt_{s''} & \defeq & \startchain{cxt_j}\chainedlink{\underline{s}_i}{\noact}\chainedlink{\noact}{\underline{s}_o}\chainend{cxt_{j+1}}.Cxt_{s}, & \mbox{ context doesn't provide $s$.} \\
%\end{array}
%\]

\subsection{Mutating entities} 
In RS, when an entity is present, it can potentially be involved in each reactions where it is required.
With a few more lines of code, in c\CNA~it is possible to describe the behaviour of a mutation of 
an entity, in a way that the mutated version of the entity can take 
part to only a subset of the reactions requiring  the  \emph{normal version} of the entity.
For example, let us assume that entity $s1$ is consumed by reactions $a1$ and $a2$.
Reaction $a1$ produces also $s1$ if $s2$ is present, otherwise $a1$ produces a mutated version of $s1$, 
say $s1'$.
When $s1'$ is produced, reaction $a2$ behaves in the same way as if $s1$ would be absent, whereas $a2$
recognises the presence of $s1'$ and behaves in the same way as if $s1$ would be present.
Technically, in both cases it is enough to add one more nondeterministic choice in the code of $P_{a1}$ and $P_{a2}$. 


\subsection{Communicating Reaction Systems}
\label{subsec:comRS}
We sketch how it is possible to program two RSs encodings, in a way that
the entities that usually come from the context of one RS will be provided instead from the other RS.

%Here we present a a simple example showing how it is possible to implement in c\CNA\ a 
%two RSs model where a subset of the entities of one RS is provided by the second one and not 
%provided by the context.

\begin{example}
\label{ex:twoRS}
Let  $rs1$ and $rs2$ be two RSs, defined, respectively, by the reactions 
%in Table~\ref{tab:rules}.
%such that $rs2$ provides the entity $s1$ to $rs1$.
$a_1=(s,\emptyset , x)$ and $a_2=(y,\emptyset  ,s)$.
%and 
%
%
%\begin{table}[t]
%\centering
%\begin{tabular}{|c|c|}
%\hline
%$rs1$  & $rs2$ \\
%\hline
%%$r1_1=(s1,\ , s2)$ & $r2_1=(s3,\  ,s1)$\\
%$a_1=(s,\emptyset , x)$ & $a_2=(y,\emptyset  ,s)$\\
%%$r1_2=(\ , s2,s1)$ & $r2_2=(s1,\ ,s3)$\\
%\hline 
%\end{tabular}
%\caption{The two Reaction Systems $rs1$ and $rs2$.}
%\label{tab:rules}
%\end{table}
% \noindent
Now, we set our example such that the two contexts, for  $rs1$ and $rs2$, do not provide any entities.
We also assume that entity $s$ in $rs1$ is provided by $rs2$, as $rs2$ produces a quantity of $s$ that is enough for $rs1$ and $rs2$.
For technical reasons, we can not use the same name for  $s$ in both the two RSs, then we use the name $ss$ in $rs2$.
%
We need to modify our encoding
%translation 
technique to suit this new setting. 
As we do not model contexts, we introduce  \emph{dummy} channel names $dx$ and $dss$ to model the absence of entities. Also, thanks to the simplicity of the example, we can leave out the use of the   $p_i$ channels. This streamlining does not affect the programming technique we propose to make two RSs communicate.
%Thus, there will be no $Cxt$ processes and we manage the code in a way that $rs1$ will receive the entity $s1$ from $rs2$. 
%Please also note that, for our simple example, we do not have anymore to specify the all steps
%of a interactive system, as the communication mechanism from $rs2$ to $rs1$ will generate an infinite
%behaviour.
%\noindent
%First, we translate the rules in $rs1$:
First, we translate the reaction in $rs1$, by setting
$\llbracket a_1\rrbracket \defeq P_{a_1}$ where:
\[
\begin{array}{lcl}
%\llbracket r1_1\rrbracket \defeq P_{r1_1}&\defeq &\startchain{\silent}\chainedlink{s1}{\noact}\chainedlink{\noact}{s1}\chainedlink{\widetilde{s2}}{\noact}\chainedlink{\noact}{\widetilde{s2}}\chainend{r1_2}.P_{r1_1} +
%\startchain{\silent}\chainedlink{\overline{s1}}{\noact}\chainedlink{\noact}{\overline{s1}}\chainend{r1_2}.P_{r1_1}\\
 P_{a_1}&\defeq &
\startchain{\silent}\chainedlink{s_i}{{\color{blue}\noact}}\chainedlink{{\color{blue}\noact}}{s_o}\chainedlink{\widetilde{x}_i}{{\color{gray}\noact}}\chainedlink{{\color{gray}\noact}}{\widetilde{x}_o}\chainend{a_2}.P_{a_1} +
\startchain{\silent}\chainedlink{\overline{s}_i}{\noact}\chainedlink{\noact}{\overline{s}_o}    \chainedlink{dx_i}{\noact}\chainedlink{\noact}{dx_o}\chainend{a_2}.P_{a_1}\\
%\end{array}
%\]
%\\
%
%\begin{array}{lcl}
%\llbracket r1_2\rrbracket \defeq P_{r1_2}&\defeq &\startchain{r1_2}\chainedlink{\overline{s2}}{\noact}\chainedlink{\noact}{\overline{s2}}\chainedlink{\widetilde{s1}}{\noact}\chainedlink{\noact}{\widetilde{s1}}\chainend{r2_1}.P_{r2_1} +
%\startchain{r1_2}\chainedlink{s2}{\noact}\chainedlink{\noact}{s2}\chainend{r2_1}.P_{r2_1}\\
\end{array}
\]

\noindent
Please note, that prefixes of process $P_{a_1}$ end with the channel name $a_2$, as the link chain is now connected with the  reaction of  $rs2$.
The encoding
%translation 
for the entities is given by setting $\llbracket s \rrbracket \defeq P_{s}$ and $\llbracket x \rrbracket \defeq  P_{x}$, where:
\[
\begin{array}{lcl}
P_{s} &\defeq & \startchain{{\color{blue}s_i}}\chainedlink{{\color{blue}s_o}}{\noact}\chainedlink{\noact}{{\color{blue}\widehat{s}_i}}\chainend{{\color{blue}\widehat{s}_o}}.P_{s} + \startchain{s_i}\chainend{s_o}.\overline{P_{s}}\\
\overline{P_{s}}&\defeq &\startchain{\overline{s}_i}\chainedlink{\overline{s}_o}{\noact}\chainedlink{\noact}{\widehat{s}_i}\chainend{\widehat{s}_o}.P_{s}+\startchain{\overline{s}_i}\chainend{\overline{s}_i}.\overline{P_{s}}\\
%\begin{array}{lclclll}
P_{x}& \defeq &\startchain{\widetilde{x}_i}\chainend{\widetilde{x}_o}.P_{x} +  \startchain{dx_i}\chainend{dx_o}.\overline{P_{x}}\\
%\\
%&&\overline{P_{s2}}&\defeq &\startchain{\overline{s2}}\chainedlink{\overline{s2}}{\noact}\chainedlink{\noact}{\widetilde{s2}}\chainend{\widetilde{s2}}.P_{s2} +
%\startchain{\overline{s2}}\chainend{\overline{s2}}.\overline{P_{s2}}\\
%\\
%\\
\overline{P_{x}}&\defeq &\startchain{{\color{gray}\widetilde{x}_i}}\chainend{{\color{gray}\widetilde{x}_o}}.P_{x} +  \startchain{dx_i}\chainend{dx_o}.\overline{P_{x}}\\
%\end{array}
%\]
%
%\[

\end{array}
\]


%\[
%\begin{array}{lclclll}
%\llbracket s2 \rrbracket &\defeq& P_{s2}&\defeq &\startchain{s2}\chainedlink{s2}{\noact}\chainedlink{\noact}{\widetilde{s2}}\chainend{\widetilde{s2}}.P_{s2} +  \startchain{s2}\chainend{s2}.\overline{P_{s2}}\\
%%\\
%&&\overline{P_{s2}}&\defeq &\startchain{\overline{s2}}\chainedlink{\overline{s2}}{\noact}\chainedlink{\noact}{\widetilde{s2}}\chainend{\widetilde{s2}}.P_{s2} +
%\startchain{\overline{s2}}\chainend{\overline{s2}}.\overline{P_{s2}}\\
%
%\end{array}
%\]


\noindent
The encoding
%translation 
for $rs2$ is given by $\llbracket a_2\rrbracket \defeq P_{a_2}$, where:

\[
\begin{array}{lcl}
P_{a_2}&\defeq &\startchain{a_2}\chainedlink{y_i}{{\color{gray}\noact}}\chainedlink{{\color{gray}\noact}}{y_o}\chainedlink{\widetilde{ss}_i}{{\color{red}\noact}}\chainedlink{{\color{red}\noact}}{\widetilde{ss}_o}\chainend{\silent}.P_{a_2}+
\startchain{a_2}\chainedlink{\overline{y}_i}{\noact}\chainedlink{\noact}{\overline{y}_o}    
\chainedlink{dss_i}{\noact}\chainedlink{\noact}{dss_o}\chainend{\silent}.P_{a_2}\\
%\end{array}
%\]
%
%\[
%\begin{array}{lcl}
%\llbracket r2_2\rrbracket \defeq P_{r2_2}&\defeq &\startchain{r2_2}\chainedlink{ss1}{\noact}\chainedlink{\noact}{ss1}\chainedlink{\widetilde{s3}}{\noact}\chainedlink{\noact}{\widetilde{s3}}\chainend{\silent}.P_{r2_2}
%+
%\startchain{r2_2}\chainedlink{\overline{ss1}}{\noact}\chainedlink{\noact}{\overline{ss1}}\chainend{\silent}.P_{r2_2}\\
\end{array}
\]
\noindent
In the encoding
%translation 
of the entities in  $rs2$, we introduce the mechanism that allows the entity
$s$ ($ss$ in $rs2$) to be provided in $rs1$. Every time $ss$ is produced in $rs2$,  a virtual link is
created to synchronise with $rs1$   on link $\link{\widehat{s}_i}{\widehat{s}_o}$. 
To this purpose we define $\llbracket ss \rrbracket \defeq P_{ss}$ and $\llbracket y\rrbracket\defeq  P_{y}$, where:

\[
\begin{array}{lcl}
P_{ss}&\defeq &\startchain{\widetilde{ss}_i}\chainedlink{\widehat{s}_i}{\noact}\chainedlink{\noact}{\widehat{s}_o}\chainend{\widetilde{ss}_o}.P_{ss}+
\startchain{dss_i}   \chainend{dss_o}.\overline{P_{ss}} \\
%\end{array}
%\]
%\[
%\begin{array}{lcllcl}
\overline{P_{ss}}&\defeq &\startchain{{\color{red}\widetilde{ss}_i}}\chainedlink{{\color{red}\widehat{s}_i}}{{\color{blue}\noact}}\chainedlink{{\color{blue}\noact}}{{\color{red}\widehat{s}_o}}\chainend{{\color{red}\widetilde{ss}_o}}.P_{ss}
+
\startchain{dss_i}\chainend{dss_o}.\overline{P_{ss}}\\
P_{y}&\defeq &\startchain{{\color{gray}y_i}}\chainend{{\color{gray}y_o}}.\overline{P_{y}} \\
\overline{P_{y}}&\defeq &
%\startchain{\overline{y}_i}\chainedlink{\overline{y}_o}{\noact}\chainedlink{\noact}{\widetilde{y}_i}\chainend{\widetilde{y}_o}.P_{y}
%+ 
\startchain{\overline{y}_i}\chainend{\overline{y}_o}.\overline{P_{y}}
\end{array}
\]
%\[
%\begin{array}{lcllcl}

%\end{array}
%\]
\end{example}

\noindent
We now assume that the initial system is $S \defeq \restrict{names}(P_{a_1}|P_{a_2}|P_s|P_y|\overline{P_x}|\overline{P_{ss}})$, i.e.
only entities $s$ and $y$ are present. Now, the only possible transition has  the following label (that we report without restriction):
\[
\startchain{\silent}\chainedlink{s_i}{{\color{blue}s_i}}\chainedlink{{\color{blue}s_o}}{s_o}\chainedlink{\widetilde{x}_i}{{\color{gray}\widetilde{x}_i}}\chainedlink{{\color{gray}\widetilde{x}_o}}{\widetilde{x}_o} \chainedlink{a_2}{a_2}\chainedlink{y_i}{ {\color{gray}y_i}}\chainedlink{{\color{gray}y_o}}{y_o}\chainedlink{\widetilde{ss}_i}{{\color{red}\widetilde{ss}_i}}
\chainedlink{{\color{red}\widehat{s}_i}}{{\color{blue}\widehat{s}_i}}
\chainedlink{{\color{blue}\widehat{s}_o}}{{\color{red}\widehat{s}_o}}
\chainedlink{{\color{red}\widetilde{ss}_o}}{\widetilde{ss}_o}\chainend{\silent},
\]
where the black links belong to the prefixes of $P_{a_1}$, and $P_{a_2}$, the blue links belong to $P_s$, the gray links belong to $P_y$, and $\overline{P_x}$ and the red links belong to $\overline{P_{ss}}$.
After the execution, the entity $s$ is still present in $rs1$ as it has been provided by $rs2$.\\

As we have briefly sketched, our model of two 
 \emph{communicating Reaction Systems} can enable the study of the behaviour of one RS in relation to another one. 
Thus, the products of the reactions of one RS can become the input for another one. 
This could allow for a modular approach to modeling complex systems, by composing different Reaction Systems. 


%which can be modified to represent pathological states as well. 


% Again, we stress the fact the verbosity of the {\tt link}-calculus code 
%should not be
% seen as a drawback, as it undergoes to automatic code translation.

