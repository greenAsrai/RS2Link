% !TEX root =  main_TCS.tex

In the area of Natural Computing, Reaction Systems (RSs) are
a qualitative abstraction inspired by the 
functioning of living cells, suitable to model the main
mechanisms of biochemical reactions.
%This model has already been applied
%and extended successfully to various areas of research. 
RSs interact with a context, and
pose challenges for modularity, compositionality, extendibility and behavioural equivalence. 
In this paper we define a modular encoding of RSs as processes
in the chained Core Network Algebra (cCNA), which is a new variant  of the  {\tt link}-calculus.
The encoding represents the behaviour of each entity separately and
preserves faithfully their features, and we prove its correctness and completeness.
%we consider a variant of the {\tt link}-calculus, which allows to model 
%multiparty interaction in concurrent systems, and show that we can
%represent reaction systems as processes 
%in a modular way, by representing the behaviour of each entity and
%preserving faithfully their features.
%We show the correctness and completeness 
%of our embedding.
%We illustrate our framework by 
%{\color{red} We show} the embedding of 
%a few examples expressing computer science
%and biological applications. 
Our encoding provides a Labelled Transition System (LTS) semantics for RSs. Based on the LTS semantics, we adapt  the classical notion of  bisimulation to define a novel equivalence, called bio-similarity, for studying properties of RSs.
%On top of the LTS semantics for reaction systems provided by our framework, 
%we then adapt the classical notion (in concurrency) of bisimulation
%to make it more suitable for studying properties of reaction systems.
In particular, we define a new assertion language based on regular expressions, 
which allows to specify the properties of interest, and use it to extend Hennessy-Milner logic
to our setting.
% framework. 
We prove that our bio-similarity relation and the logical equivalence, that are defined parametrically on some assertion of interest, coincide.
%Interestingly, the novel notion of bisimilarity and logical equivalence that are defined parametrically on some assertion of interest are proved to coincide, like in the classical case.
Finally, 
we claim that
our encoding 
%can 
contributes to increase the expressiveness
of RSs, by exploiting the interaction among 
different RSs. 
