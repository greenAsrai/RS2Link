% !TEX root =  main_TCS.tex

In the area of Natural Computing, reaction systems are
a qualitative abstraction inspired by the 
functioning of living cells, suitable to model the main
mechanisms of biochemical reactions.
This model has already been applied
and extended successfully to various areas of research. Reaction 
systems interact with the environment represented by the context, and
pose challenges for further extensions as well as
for the implementation, as it is a new computation model. 
In this paper {\color{red} we define a modular embedding of Reaction Systems as processes
in the chained Core Network Algebra (cCNA), which is a variant  of the  {\tt link}-calculus,
by representing the behaviour of each entity and
preserving faithfully their features, and we prove its correctness and completeness.}
%we consider a variant of the {\tt link}-calculus, which allows to model 
%multiparty interaction in concurrent systems, and show that we can
%represent reaction systems as processes 
%in a modular way, by representing the behaviour of each entity and
%preserving faithfully their features.
%We show the correctness and completeness 
%of our embedding.
%We illustrate our framework by 
{\color{red} We show} the embedding of 
a few examples expressing computer science
and biological applications. 
{\color{red}Our embedding provides a Labelled Transition System (LTS) semantics for reaction systems. Based on the LTS semantics, we adapt  the classical notion of  process bisimulation to define a bio-simulation relation for studying properties of reaction systems.}
%On top of the LTS semantics for reaction systems provided by our framework, 
%we then adapt the classical notion (in concurrency) of bisimulation
%to make it more suitable for studying properties of reaction systems.
In particular, we define a new assertion language based on regular expressions, 
which allows to specify the properties of interest, and use it to extend Hennessy-Milner logic
to our {\color{red} embedding.}
% framework. 
{\color{red} We prove that our bio-simulation relation and the logical equivalence, that are defined parametrically on some assertion of interest, coincide.}
%Interestingly, the novel notion of bisimilarity and logical equivalence that are defined parametrically on some assertion of interest are proved to coincide, like in the classical case.
Finally, 
{\color{red} we claim that}
our methodology 
%can 
contribute to increase the expressiveness
of reaction systems, by exploiting the interaction among 
different reaction systems. 
